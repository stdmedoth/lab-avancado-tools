\documentclass[12pt, letterpaper]{article}
\usepackage[utf8]{inputenc}
\usepackage[portuguese]{babel}
\usepackage{graphicx}
\usepackage{amsmath}
\usepackage{geometry}
\usepackage{caption}
\usepackage{pgfplots}
\usepackage{ulem}
\usepackage{circuitikz}
\usepackage{float}
\pgfplotsset{width=10cm, compat=1.17}
\geometry{
	paper=a4paper, 
	inner=3cm,
	outer=3cm,
	bindingoffset=.5cm, 
	top=2cm, 
	bottom=2cm, 
}
\graphicspath{ {./images/} }
\newcommand{\source}[1]{\caption*{Fonte: {#1}} }

\begin{document}

\begin{titlepage}
    \begin{center}
        \Large
        Universidade de São Paulo
        \vspace*{1cm}

        \includegraphics[scale=0.75]{images/ifsc_logo.png} \\
        \Huge
        \textbf{[Atividade]} \\
        \huge
        [Título]
        
        \Large
        [Disciplina]

        \vspace{4cm}
        \Large
        
    \end{center}
    
    \begin{flushright}
        \Large
        \textbf{Aluno [ou Alunos]}\\
        \large
            [Seu nome] n° [Seu número usp]
    \end{flushright}
    
    \vspace*{\fill}
    \centering \large São Carlos \\ 2024
    
\end{titlepage}

\section{Objetivos}

\section{Introdução}
\subsection{Aqui, abra quantas subseções forem necessárias para introduzir o conteúdo teórico de seu Relatório}

\section{Metodologia e Resultados}
\subsection{[Aqui, abra quantas subseções forem necessárias para elaborar sobre todos os assuntos de seu Relatório]}


\begin{figure}[h!]
    \centering
    % Aqui você puxa a imagem única que o Python gerou
    \includegraphics[width=0.8\textwidth]{images/fit_plot.png}
    
    % A legenda explica as duas partes
    \caption{
        Ajuste linear dos dados experimentais de Voltagem vs. Frequência. 
        \textbf{Superior:} Os pontos pretos indicam os dados com barras de erro e a linha vermelha o modelo ajustado ($V = af + b$). 
        \textbf{Inferior:} Gráfico de resíduos ($Y_{exp} - Y_{modelo}$). A distribuição aleatória em torno de zero indica um bom ajuste estatístico.
    }
    \label{fig:analise_completa}
\end{figure}
\section{Conclusão}


\end{document}